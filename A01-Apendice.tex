\chapter{Validación del programa \textit{dtkintegrate}}
    \section{Eficiencia de \textit{dtkintegrate}}
        Para corroborar que el programa \textit{dtkintegrate} presenta buena precisión, se calcularon los errores relativos de las integrales de las densidades electrónicas en los espacios de posiciones y momentos para cada uno de los átomos (1-36) y se compararon contra su límite de redondeo a la unidad más cercana (figura \ref{fig:relErrRhoLV}). Cuando el error relativo calculado es mayor a su límite de redondeo, el redondeo de la integral arrojará un valor distinto al correcto.
        
        \begin{figure}[thbp]
            \centering
            \includegraphics[width=\textwidth]{images/relErrRhoLV.pdf}
            \caption{Error relativo de la integral de la densidad electrónica en los espacios de posiciones (cículos) y momentos (cuadrados) contra el número atómico (1-36). La línea punteada representa el límite de redondeo a la unidad más cercana. Se usó un total de 200,000 puntos de muestreo.}
            \label{fig:relErrRhoLV}
        \end{figure}
        
        Los parámetros utilizados para la integración fueron:
        \begin{itemize}
            \item Número total de puntos de muestreo: 200,000.
            \item Número de intervalos: 10.
            \item Máximo número de iteraciones: 20.
            \item Taza de convergencia: 1.
            \item Termalización: 0.
            \item No detener refinamiento de la malla durante la integración.
            \item Tolerancia: 0.
        \end{itemize}
        
        La figura \ref{fig:relErrRhoLV} muestra que los valores obtenidos son suficientemente precisos para el cálculo del número de electrones a partir de densidades electrónicas, a pesar de ser un método aleatorio, haber utilizado 200,000 puntos y presentar fluctuaciones alrededor del valor exacto, pues estas no superan el límite de redondeo.
        
    \section{Eficiencia de \textit{dtkintegrate} contra el método Monte Carlo}
        También se puede probar la eficiencia del programa. Para eso, se desarrolló e implementó el algoritmo de integración Monte Carlo en \textit{DensToolKit} y se compararon ambos programas, tomando como referencia el porcentaje de error relativo contra el número de evaluaciones sobre el integrando. La función a integrar fue la densidad electrónica del metano, cuyo número de electrones es 10 y el resto de los parámetros del programa \textit{dtkintegrate} se mantuvieron fijos en los establecidos por defecto:
        \begin{itemize}
            \item Taza de convergencia: 1.
            \item Número de intervalos: 10.
            \item Máximo número de iteraciones: 20.
            \item Termalización: 0.
            \item No detener refinamiento de la malla durante la integración.
            \item Tolerancia: 0.
        \end{itemize}
        
        Se puede apreciar en la figura \ref{fig: mc_vs_lv} que los resultados obtenidos por Las Vegas+ son notablemente más acertados que aquellos por Monte Carlo, sin importar el número de evaluaciones que se realicen (puntos de muestreo) sobre el integrando.
        
        \begin{figure}[htbp]
            \centering
            \includegraphics[width=\textwidth]{images/MC-LV.jpg}
            \caption{Comparación del porcentaje de error relativo contra número de evaluaciones sobre la integral de la densidad electrónica del metano (10 electrones). Los métodos de integración usados fueron Monte Carlo (azul) y Las Vegas+ (naranja). Se usaron entre 10,000 y 1,000,000 evaluaciones sobre el integrando.}
            \label{fig: mc_vs_lv}
        \end{figure}
    
    \section{Entropías de Shannon en átomos utilizando \textit{dtkintegrate}}
        \noindent Para corroborar el funcionamiento del programa implementado en \textit{DensToolKit}, se calcularon las entropías informacionales de las densidades normalizadas al número de electrones y a la unidad, así como de la densidad relativa, la cual toma como referencia su máximo global. La entropía se integró en átomos simples con números atómicos desde 1 hasta 36, utilizando como combinaciones método-base MP2/6-31+G* y HF/cc-pvTZ, ambos obtenidos por GAUSSIAN09 \cite{gaussian09}. Los resultados se compararon con aquellos publicados por \textit{Flores-Gallegos, N.} \cite{flores_gallegos} y \textit{Hô, M. et al.} \cite{flores_gallegos,sagar}, quienes realizaron los mismos cálculos y con los mismos átomos, pero con combinaciones método-base CISD(full)/cc-pvTZ y HF/6-31G**, respectivamente. Los resultados mostrados en las figuras \ref{fig: entropies 1-36} resultaron ser muy semejantes a aquellos publicados por los autores mencionados anteriormente. Sin embargo, las pequeñas diferencias se deben al uso de diferentes bases o métodos.
        
        \begin{figure}
            \centering
            \begin{subfigure}[htbp]{0.774\textwidth}
                \centering
                    \includegraphics[width=\textwidth]{images/entropies-N-1_36.jpg}
                    \caption{Entropías de Shannoon sobre átomos desde el Hidrógeno hasta el Criptón. La densidad electrónica está normalizada al número de electrones del átomo respectivo y se muestra en los espacios de posiciones (cuadrado azul) y momentos (triángulo naranja). Además, se presenta la suma de ambas entropías (círculo verde).}
                    \label{fig: entropies N 1-36}
            \end{subfigure}
            \begin{subfigure}[htbp]{0.774\textwidth}
                    \centering
                    \includegraphics[width=\textwidth]{images/entropies-1-1_36.jpg}
                    \caption{Entropías de Shannoon sobre átomos desde el Hidrógeno hasta el Selenio. La densidad electrónica está normalizada a la unidad y se muestra en los espacios de posiciones (cuadrado azul) y momentos (triángulo naranja).}
                    \label{fig: entropies 1 1-36}
            \end{subfigure}
            \begin{subfigure}[htbp]{0.774\textwidth}
                \centering
                \includegraphics[width=\textwidth]{images/entropies-Rel-1_36.jpg}
                \caption{Entropías de Shannoon sobre átomos desde el Hidrógeno hasta el Selenio. La densidad electrónica es relativa y toma como referencia el valor máximo de la densidad de cada átomo. También se muestra en los espacios de posiciones (cuadrado azul) y momentos (triángulo naranja).}
                    \label{fig: entropies Rel 1-36}
                \end{subfigure}        
                \caption{}
                \label{fig: entropies 1-36}
            \end{figure}
            
        Los resultados mencionados anteriormente muestran que el programa \textit{dtkintegrate} se implementó adecuadamente en la suite \text{DensToolKit} y muestran la eficiencia esperada del algoritmo Las Vegas+ en funciones de comportamiento gaussiano.
    