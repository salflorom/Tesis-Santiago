\addcontentsline{toc}{chapter}{Resumen}
\chapter*{Resumen}
    \noindent En este trabajo se estudia la entropía de la información de Shannon en un conjunto de átomos en su estado base. Se busca establecer la posible aplicación de la entropía de Shannon como una medida de la calidad de la descripción de la densidad electrónica. Para esto, se calculan las densidades electrónicas de los átomos utilizando diversas bases y métodos de la química computacional. La entropía de Shannon se obtiene a través de una integración numérica con el método Las Vegas+. Parte importante del trabajo consiste en la implementación eficiente de este algoritmo, además de adaptarlo a la suite \textit{DensToolKit}, el cual está programado en C++ y es un código libre.
    
\chapter{Introducción}
    \noindent Dos de los conceptos medulares en la química computacional son la función de onda y la densidad electrónica, pues estas dos albergan toda la información electrónica del sistema, ya sea átomo o molécula, y el conocimiento de su comportamiento electrónico permite describir las causas y consecuencias de la formación y rompimiento de enlaces. Incluso, su importancia ha incitado a que varios autores hayan desarrollado programas o paquetes, como \textit{DensToolKit}, que se enfocan en calcular la densidad electrónica de un sistema específico y realizar diversos análisis con ella misma.
    
    Uno de los problemas presentes en la química computacional es que la única manera hasta ahora conocida de aproximarse a la función de onda de un átomo o molécula es a través de métodos numéricos como Hartree-Fock (HF) o los métodos perturbativos M\o ller-Plesset, los cuales son MP2(full), MP3(full), MP4(SDQ), entre otros. La manera en que se utilizan los métodos para describir el comportamiento electrónico de átomos es primero componiendo funciones de onda llamadas orbitales moleculares, conformadas por combinaciones lineales de funciones monoelectrónicas llamadas orbitales atómicos que aproximen el estudio del sistema a una superposición de diversos átomos hidrogeoides, cuyas soluciones pueden conocerse al resolver la ecuación de Schr\"odinger para cada orbital mencionado. Una manera de componer tales orbitales atómicos es a través de combinaciones lineales de funciones llamadas funciones primitivas y algunos ejemplos de estos son 6-31G, 6-31+G**, 6-31G(2df,p), aug-cc-pVDZ o aug-cc-pVTZ, a los cuales uno puede acceder a través de programas como Gaussian09 \cite{gaussian09}. Estos son conjuntos de funciones de tipo gaussiano que al superponerse se asemejan a los orbitales atómicos.
    
    Respecto a la densidad electrónica, su utilización surge por el desarrollo de la Teoría del Funcional de la Densidad (DFT por sus siglas en inglés), cuyo problema radica en el desconocimiento del funcional de la energía de correlación e intercambio ($E_{CX}$) en la mayoría de los sistemas conocidos. No obstante, varios autores han sugerido aproximaciones a partir del acompañamiento de parámetros experimentales y de sistemas conocidos como un gas de electrones uniforme, donde este funcional sí se conoce y a volúmenes infinitesimales su comportamiento se asemeja a una mayor variedad de sistemas. Los funcionales B3LYP y $\omega$B97X-D presentan esas características y también son accesibles a través de Gaussian09. Respecto a la densidad electrónica, esta puede calcularse a través de los mismos conjuntos de funciones base mencionados en el párrafo anterior. Cabe mencionar que los funcionales en DFT habrán de depender esencialmente de la densidad electrónica.
    
    Además del funcional de la energía de correlación e intercambio, existe una gran variedad de funcionales y uno de ellos es la entropía de Shannon, la cual también llamaremos en la presente como entropía informacional. Este objeto es de interés en la química computacional debido a varias razones, pero en la tesis presente, el enfoque se encuentra en que ha sido sugerida para medir la calidad de los conjuntos de funciones base, haciendo referencia a <<calidad>> como las aproximaciones de los efectos de la función de onda calculada numéricamente con respecto a las observaciones experimentales. A pesar de que varios autores han estado estudiando esta utilidad \cite{sagar,ho,tripathi,gadre}, hasta ahora no se ha logrado concluir si esta es determinante para la medida de la calidad mencionada.
    
    Por último, es importante mencionar que la teoría DFT tiene la dificultad de que la densidad electrónica presenta desde uno a varios centros cuasi-singulares que complican su integración y hay funcionales que no son fáciles de conocer hasta el presente. Por consiguiente, no sería fácil utilizarlos como integrandos, tal como sucede con la entropía de Shannon. Entonces se vuelve necesario el uso de métodos de integración que no requieran previo conocimiento del comportamiento del integrando, como el método de integración Monte Carlo o sus derivados como Las Vegas+, método que se basa principalmente en el muestreo de importancia.
    
    En la tesis presente se mencionará la implementación del algoritmo de integración Las Vegas+ en el paquete de programas \textit{DensToolKit} para probar si la entropía informacional es útil para medir la calidad de conjuntos de funciones base, de los cuales se utilizaron los mencionados al principio de este capítulo. Los sistemas de estudio fueron el conjunto de átomos del hidrógeno al kriptón y los métodos y funcionales utilizados también fueron los mencionados arriba.
    