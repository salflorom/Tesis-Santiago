\chapter{Resumen}
    \noindent En este trabajo se estudia la entropía de la información de Shannon en un conjunto de átomos y moléculas en su estado base. Se busca establecer la posible aplicación de la entropía de Shannon como una medida de la calidad de la descripción de la densidad electrónica de los átomos. Para esto, se calculan las densidades electrónicas de los átomos y moléculas utilizando diversas bases y métodos de la química computacional. La entropía de Shannon se obtiene a través de una integración numérica con el método Vegas-Monte Carlo. Parte importante del trabajo consiste en la implementación eficiente de este algoritmo, además de adaptarlo a la suite \textit{DensToolKit}, el cual está programado en C++ y es un código libre.
    
\chapter{Introducción}
    \noindent El uso de la entropía de Shannon o entropía informacional en el campo de la química computacional tiene sus bases en los estudios de varios investigadores, tales como \textit{Brillouin, L.} \cite{brillouin1951}, quien asociaba la entropía informacional con la de Boltzmann. Se llegó incluso a deducir, por ejemplo, por \textit{Chakrabarti} \cite{chakrabarti}, que esta última es un caso especial de la entropía de Shannon debido a que todos los estados posibles que puede presentar un sistema tienen la misma probabilidad de ocurrencia.
    
    A partir de las analogías descubiertas, \textit{Bia{\l}ynicki-Birula, I. y Mycielski, J.} \cite{bialynicki} la usaron para expresar una forma análoga de la incertidumbre de Heisenberg, reemplazando las desviaciones estándar por las entropías, manteniendo un significado físico equivalente al de la incertidumbre de Heisenberg.
    
    Desde entonces, autores como \textit{Gadre, S. et al.} \cite{gadre} han propuesto que la entropía de Shannon podía tener alguna utilidad en cuanto a la calidad de las bases o métodos usados para derivar funciones de onda de átomos o moléculas, refiriéndonos a <<calidad>> como la aproximación numérica que tiene una función de onda con respecto a las observaciones experimentales. Siendo este hecho el inicio del uso de esta herramienta en el campo de la química computacional, varios autores empezaron a estudiarla con mayor detalle \cite{sagar,ho,tripathi}, profundizando su aplicación en la calidad anteriormente mencionada y en su relación con la energía del sistema. Sin embargo, hasta ahora, no hay suficiente información para concluir si la entropía de Shannon es determinante para medir la calidad de una base.
    
    En la tesis presente, se empleó la suite \textit{DensToolKit} e implementamos en ella el algoritmo de integración Vegas-Monte Carlo para probar la entropía informacional en diversos conjuntos base gaussianos, usando el método Hartree-Focky diversos métodos post-Hartree-Fock en varios átomos (del hidrógeno al criptón) y moléculas (ácido fluorhídrico, agua, amoniaco y metano) para corroborar si se presenta una relación entre esta cantidad y tales bases o métodos.