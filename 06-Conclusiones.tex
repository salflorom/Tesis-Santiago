\chapter{Conclusiones}
    \noindent Se desarrolló e implementó el algoritmo Las Vegas+ para funciones 3D sobre el paquete de programas \textit{DensToolKit}, permitiendo utilizar tal algoritmo en átomos para analizar sus propiedades. Una vez implementado en \textit{DensToolKit}, se analizó su eficiencia comparando sus cálculos con aquellos realizados por el método Monte Carlo, así como con aquellos publicados por Flores-Gallegos, N. \cite{flores_gallegos} y Hô, M. et al. \cite{sagar}. Se probó que el método implementado presenta una eficiencia muy por encima de Monte Carlo, sus cálculos mostraron buena precisión con no más de 200,000 puntos de muestreo y fueron comparables con los realizados por los mismos autores.
    
    Se calcularon las entropías de Shannon sobre 36 átomos crecientes en número atómico y con métodos y bases distintos para corroborar una relación entre la entropía de Shannon y la calidad de las bases. Los resultados mostraron una aparente relación entre la entropía y los métodos y bases utilizados en el presente trabajo al notar fluctuaciones alrededor de la base 6-31 en las figuras \ref{fig: deltasum-n-tot-1} y \ref{fig: deltasum-n-tot-2} que van incrementando conforme crece el número atómico, sin embargo, no parece fungir adecuadamente como medida de la calidad de las bases, pues todas estas mostraban ser mejores para algunos átomos y peores para otros. Cabe mencionar que tales aparentes fluctuaciones se comportaron de forma distinta entre todas ellas, lo cual también mostró una dependencia entre método (o funcional) y base, además de que el método también influye en la entropía.
    
    Se tienen esperados más trabajos para realizar análisis más detallados sobre la calidad de las bases a través de la entropía. Se propone comparar las entropías informacionales con las energías de los sistemas formados por distintas bases y calculadas por distintos métodos para determinar si hay una relación entre la entropía y la energía calculada numéricamente. También se pretende realizar un estudio semejante con las entropías informacionales propuestas por otros autores como Flores-Gallegos, N. \cite{flores_gallegos} u otros y realizar el mismo estudio sobre moléculas para determinar si se cumplen comportamientos semejantes al de los átomos. Por último, dado que los métodos y funcionales influyen en la entropía informacional de átomos, se propone realizar un estudio donde se comparen diferentes métodos o funcionales por cada base.