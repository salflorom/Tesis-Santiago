\chapter{Conclusiones}
    \noindent Se desarrolló e implementó el algoritmo Las Vegas-Monte Carlo para funciones 3D sobre el paquete de programas \textit{DensToolKit}, permitiendo utilizar tal algoritmo en átomos y moléculas para analizar sus propiedades. Una vez implementado en \textit{DensToolKit}, se analizó su eficiencia comparando sus cálculos con aquellos realizados por el método Monte Carlo, así como con aquellos publicados por \textit{Flores-Gallegos, N.} \cite{flores_gallegos} y \textit{Hô, M. et al.} \cite{flores_gallegos,sagar}. Se probó que el método implementado presenta una eficiencia muy por encima de Monte Carlo y realiza cálculos comparables con los realizados por los mismos autores.
    
    Se calcularon las entropías de Shannon sobre 36 átomos crecientes en número atómico y 5 moléculas isoelectrónicas a 10 electrones, ambos casos en métodos y bases distintos para corroborar una relación entre la entropía de Shannon y la calidad de las bases. Los resultados mostraron una aparente relación entre la entropía y los métodos y bases utilizados en el presente trabajo sólo cuando se tratan todos los átomos en conjunto y no de forma individual, pues se descubrió que la entropía de Shannon no parece fungir adecuadamente en la medida de la calidad de las bases cuando se compara entre átomos, sin embargo, mostraron indicios de fungir como una medida adecuada de la calidad en bases y métodos al considerar una entropía sobre todos los átomos en general y no sobre átomos en específico.
    
    Se tienen esperados más trabajos para realizar análisis más detallados sobre la calidad de las bases y métodos a través de la entropía. Se propone comparar las entropías informacionales con las energías de los sistemas formados por distintas bases y calculados por distintos métodos.