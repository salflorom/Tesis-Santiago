\chapter{Conclusiones}
    \noindent Se desarrolló e implementó el algoritmo Las Vegas-Monte Carlo para funciones 3D sobre el paquete de programas \textit{DensToolKit}, permitiendo utilizar tal algoritmo en átomos para analizar sus propiedades. Una vez implementado en \textit{DensToolKit}, se analizó su eficiencia comparando sus cálculos con aquellos realizados por el método Monte Carlo, así como con aquellos publicados por \textit{Flores-Gallegos, N.} \cite{flores_gallegos} y \textit{Hô, M. et al.} \cite{flores_gallegos,sagar}. Se probó que el método implementado presenta una eficiencia muy por encima de Monte Carlo, sus cálculos mostraron buena precisión con no más de 200,000 puntos de muestreo y fueron comparables con los realizados por los mismos autores.
    
    Se calcularon las entropías de Shannon sobre 36 átomos crecientes en número atómico y con métodos y bases distintos para corroborar una relación entre la entropía de Shannon y la calidad de las bases. Los resultados mostraron una aparente relación entre la entropía y los métodos y bases utilizados en el presente trabajo al notar fluctuaciones alrededor de la base 6-31 en las figuras \ref{fig: deltasum-n-tot-1} y \ref{fig: deltasum-n-tot-2} que van incrementando conforme incrementa el número atómico, sin embargo, no parece fungir adecuadamente como medida de la calidad de las bases, pues todas las bases mostraban se mejores para algunos átomos y peores para otros. Sin embargo, todas las bases dependen de parámetros variacionales que se ajustan al sistema, razón por la cual es posible que en algunos casos las bases muestren mejoría, pero en otros no.
    
    Se tienen esperados más trabajos para realizar análisis más detallados sobre la calidad de las bases a través de la entropía. Se propone comparar las entropías informacionales con las energías de los sistemas formados por distintas bases y calculados por distintos métodos para determinar si hay una relación entre la entropía y la energía calculada por métodos variacionales. También se pretende realizar un estudio semejante con las entropías informacionales propuestas por otros autores como \textit{Flores-Gallegos, N.} \cite{flores_gallegos} u otros. Por último, se propone realizar el mismo estudio sobre moléculas para determinar si se cumplen comportamientos semejantes al de los átomos.